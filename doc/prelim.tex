\documentclass{sig-alternate}

\usepackage{algorithm}
\usepackage{algpseudocode}
\usepackage{listings}

\usepackage{tikz}
\usepackage{pgf-umlsd}
\usepgflibrary{arrows} % for pgf-umlsd

\usepackage{bytefield}

\usepackage{fourier}
\usepackage{enumitem}

\begin{document}

\title{High-Level Sprout Geometry Extraction}
\author{
	Gio Carlo Cielo Borje \\
	\affaddr{gborje@uci.edu} \\
	\affaddr{University of California, Irvine}
}
\date{\today}
\maketitle

\section{Introduction} % (fold)
\label{sec:Introduction}

The objective of this program is to extract the high-level geometric features
of blood vessel in pictures. The high-level geometry of a blood vessel
comprises of a bead and several sprouts.

In addition to the High-Level Sprout Geometry (HLSG) Extractor, a driver is
implemented to perform feature extraction on sample images and analyze the
extracted features.

% section Introduction (end)

\section{Data Structures} % (fold)
\label{sec:Data Structures}

The following data structures are used to implement the HLSG Extractor.

	\subsection{Bead Feature} % (fold)
	\label{sub:Bead Feature}

	A bead feature is an abstraction of the transverse view of the primary
	blood vessel from which other blood vessels sprout from. The geometry
	of the bead is described by the descriptor below.
	\begin{figure}[!ht]
		\centering
		\begin{bytefield}{24}
			\bitbox{24}{\textbf{Bead}} \\
			\bitbox{24}{center} \\
			\bitbox{24}{radius} \\
			\bitbox{24}{sprouts}
		\end{bytefield}
		\caption{Bead Descriptors}
		\label{fig:beaddesc}
	\end{figure}

	% subsection Bead Feature (end)

	\subsection{Sprout Feature} % (fold)
	\label{sub:Sprout Feature}

	A sprout feature is an abstraction of the blood vessels that sprout from a
	designated bead. Subsequently, sprout feature extraction is dependent upon
	bead descriptors.
	\begin{figure}[!ht]
		\centering
		\begin{bytefield}{24}
			\bitbox{24}{\textbf{Sprout}} \\
			\bitbox{24}{bead} \\
			\bitbox{24}{centroid} \\
			\bitbox{24}{length} \\
			\bitbox{24}{width} \\
			\bitbox{24}{branches}
		\end{bytefield}
		\caption{Sprout Descriptors}
		\label{fig:sproutdesc}
	\end{figure}
	
	% subsection Sprout Feature (end)

	\subsection{Driver} % (fold)
	\label{sub:Driver}

	The Driver is responsible for parsing input from the client and emulating
	the encoded actions as functions of the HLSG Extractor. That is, the Driver
	acts similar to a REPL (Read-Eval-Print-Loop) that reads input from the
	client, evaluates the input and prints the corresponding output in a loop.
	The set of commands available to the client is outlined Table
	\ref{tab:commands}.
	
	% subsection Driver (end)

% section Data Structures (end)

\section{System Architecture} % (fold)
\label{sec:System Architecture}

\begin{figure*}[t]
	\begin{centering}
		\centering
\begin{sequencediagram}
\newinst[1]{client}{User}{}
\newinst[1]{driver}{Driver}{}
\newinst[2]{ex}{Extractors}{}
\newinst[3]{anal}{Sholl Analysis}{}
\newinst[1]{reportgen}{Report Generator}{}

\begin{mess}{client}{images}{driver}{}
	\begin{sdblock}{Main Loop}{}
		\begin{call}{driver}{ExtractHLSGs(img)}{ex}{hlsg}
			\begin{callself}{ex}{ExtractBeads(img)}{beads}
			\end{callself}

			\begin{sdblock}{Sprout Loop}{}
				\begin{callself}{ex}{ExtractSprouts(img, bead)}{sprouts}
				\end{callself}
			\end{sdblock}
		\end{call}

		\begin{call}{driver}{Analyze(hlsg)}{anal}{analysis}
		\end{call}
	\end{sdblock}

	\begin{call}{driver}{GenerateReport(analyses)}{reportgen}{report}
	\end{call}
\end{mess}
\end{sequencediagram}

	\end{centering}
\end{figure*}


\begin{enumerate}[noitemsep]
	\item A
\end{enumerate}

\begin{table}
	\begin{tabular}{| l | l | p{4.5cm} |}
		\hline
		\textbf{Command} & \textbf{Output} & \textbf{Description} \\\hline
		extract [file]] & HLSG of file & Extracts the HLSG of the given file. \\\hline
		extract [files] & HLSG of files & Extacts the HLSGs of the given files. \\\hline
		exit & Goodbye & Exits the program. \\\hline
	\end{tabular}
	\label{tab:commands}
	\caption{Commands}
\end{table}

% section System Architecture (end)

\section{Pseudo Code} % (fold)
\label{sec:Pseudo Code}

This section outlines the pseudo-code for the Driver and HLSGExtractor
operations. 

	\subsection{Driver} % (fold)
	\label{sub:Driver}

	The following pseudo-code outlines the Driver which reads input from the
	client, evaluates the input as a command, prints the output as a
	consequence of executing the command and then repeats this sequence of
	operations.
	\begin{algorithm}[ht!]
		\caption{Driver}
		\begin{algorithmic}
			\Procedure{Driver}{}
				\State running $\gets$ True
				\While{running}
					\State input $\gets$ read\_input()
					\State command $\gets$ parse(input)
					\State output $\gets$ HLSGExtractor.execute(command)
					\State print(output)
				\EndWhile
			\EndProcedure
		\end{algorithmic}
	\end{algorithm}

	Note that the driver executes while the \lstinline$running$ flag is true.
	Consequently, the File System is responsible for setting this flag false.

	% subsection Driver (end)

% section Pseudo Code (end)

\end{document}
