\documentclass{sig-alternate}

\usepackage{algorithm}
\usepackage{algpseudocode}
\usepackage{listings}

\usepackage{tikz}
\usepackage{pgf-umlsd}
\usepgflibrary{arrows} % for pgf-umlsd

\usepackage{bytefield}

\usepackage{fourier}
\usepackage{enumitem}

\begin{document}

\title{High-Level Sprout Geometry Extraction for Unstained Assays of In Vitro Angiogenesis}
\author{
	Gio Borje and Craig Steinke \\
	\affaddr{University of California, Irvine}
}
\date{\today}
\maketitle

\begin{abstract}
	We have developed an automated image analysis system for the quantitative
	analysis of unstained assays of \emph{in vitro} angiogenesis. The system is
	designed for fibrin gel bead sprouting assays. The quantification system
	provides the number of primary sprouts, average branching factor and
	average length for each bead in an imaged assay.
\end{abstract}

\section{Introduction} % (fold)
\label{sec:Introduction}
	Angiogenesis is a mechanism for the formation of new blood vessels from
	pre-existing vessels. Additionally, angiogenesis is part of a critical
	phase in of solid tumor growth; tumor growth is stunted to
	approximately 1-2mm in diameter without new blood vessel supply.
	Consequently, to assess the impact of angiogenic and antiangiogenic
	agents in assays, a reliable and automated system is necessary for
	quantitative analysis.

	The system is designed to detect features, restore features and analyze
	the features of imaged assays. The high-level geometry of these assays
	comprise of many beads with several associated sprouts.

	In addition to the High-Level Sprout Geometry (HLSG) Extractor, a
	driver and report generator are implemented to drive functionality on
	sample images and generate reports on the analyses respectively.
% section Introduction (end)

\section{Data Structures} % (fold)
\label{sec:Data Structures}
	The following data structures are used to implement the HLSG Extractor.

	\subsection{Bead Feature} % (fold)
	\label{sub:Bead Feature}
		A bead feature is an abstraction of the physical bead coated with
		endothelial cells in the assay. The geometry of the bead is intuitively
		circular; subsequently the geometry can is described by the descriptor
		in Figure \ref{fig:beaddesc}.
		\begin{figure}[!ht]
			\centering
			\begin{bytefield}{24}
				\bitbox{24}{\textbf{Bead}} \\
				\bitbox{24}{center: (int, int)} \\
				\bitbox{24}{radius: int} \\
				\bitbox{24}{sprouts: [Sprout]}
			\end{bytefield}
			\caption{Bead Descriptors}
			\label{fig:beaddesc}
		\end{figure}
	% subsection Bead Feature (end)

	\subsection{Sprout Feature} % (fold)
	\label{sub:Sprout Feature}
		A sprout feature is an abstraction of the blood vessels that
		develop through angiogenesis from the designed bead. Subsequently,
		sprout feature extraction is dependent upon
		bead descriptors.
		\begin{figure}[!ht]
			\centering
			\begin{bytefield}{24}
				\bitbox{24}{\textbf{Sprout}} \\
				\bitbox{24}{bead: Bead} \\
				\bitbox{24}{centroid: (int, int)} \\
				\bitbox{24}{length: int } \\
				\bitbox{24}{width: int} \\
				\bitbox{24}{segments: [RadialSegment]}
			\end{bytefield}
			\caption{Sprout Descriptors}
			\label{fig:sproutdesc}
		\end{figure}
		The sprout is actually comprised of a set of pixel segments because
		of the possibility that a sprout is disconnected.
	% subsection Sprout Feature (end)

	\subsection{Radial Line Segment} % (fold)
	\label{sub:Radial Line Segment}
		Due to the disconnectivity of sprouts, individual sprout segments
		are represented by a radially defined line segment. That is, we
		distinguish the end points from its radial distance from the origin
		of its corresponding bead. Given a line segment, we say that and
		end point is the \emph{inner point} if it is radially closer than
		its complementary end point; otherwise, we call the end point the
		\emph{outer point}.

		In addition to the distinguishable end points, a radial line
		segment is a line fit onto a corresponding blob of pixels which can
		be considered a sprout segment.
	% subsection Radial Line Segment (end)

	\subsection{Driver} % (fold)
	\label{sub:Driver}
		The Driver is responsible for parsing input from the client and emulating
		the encoded actions as functions of the HLSG Extractor. That is, the Driver
		acts similar to a REPL (Read-Eval-Print-Loop) that reads input from the
		client, evaluates the input and prints the corresponding output in a loop.
		The set of commands available to the client is outlined Table
		\ref{tab:commands}.
	% subsection Driver (end)
% section Data Structures (end)

\section{System Architecture} % (fold)
\label{sec:System Architecture}
	The system requires Python version 2.7x with the SimpleCV package. The
	architecture of the system is based on our methodology for
	quantitatively analyzing in vitro angiogenesis. The system, however,
	incorporates modules for driving batch processes as well as a
	Read-Eval-Print-Loop (REPL) for console interaction. Finally, a module
	incorporated for generating CSV reports of the analysis. The sequence
	diagram for the system components are shown in Figure
	\ref{fig:sysarch}.
	\begin{figure*}[t!]
		\begin{centering}
			\centering
\begin{sequencediagram}
\newinst[1]{client}{User}{}
\newinst[1]{driver}{Driver}{}
\newinst[2]{ex}{Extractors}{}
\newinst[3]{anal}{Sholl Analysis}{}
\newinst[1]{reportgen}{Report Generator}{}

\begin{mess}{client}{images}{driver}{}
	\begin{sdblock}{Main Loop}{}
		\begin{call}{driver}{ExtractHLSGs(img)}{ex}{hlsg}
			\begin{callself}{ex}{ExtractBeads(img)}{beads}
			\end{callself}

			\begin{sdblock}{Sprout Loop}{}
				\begin{callself}{ex}{ExtractSprouts(img, bead)}{sprouts}
				\end{callself}
			\end{sdblock}
		\end{call}

		\begin{call}{driver}{Analyze(hlsg)}{anal}{analysis}
		\end{call}
	\end{sdblock}

	\begin{call}{driver}{GenerateReport(analyses)}{reportgen}{report}
	\end{call}
\end{mess}
\end{sequencediagram}

		\end{centering}
		\caption{High-Level Architecture}
		\label{fig:sysarch}
	\end{figure*}

	The REPL module controls the interaction between the user and the
	system. Commands available in the REPL are shown in Table
	\ref{tab:commands}.
	\begin{table}[h!]
		\begin{tabular}{| l | l | p{4cm} |}
			\hline
			\textbf{Command} & \textbf{Output} & \textbf{Description} \\\hline
			extract [file] & HLSG of file & Extracts the HLSG of the given file. \\\hline
			extract [files] & HLSG of files & Extacts the HLSGs of the given files. \\\hline
			exit & Goodbye & Exits the system. \\\hline
		\end{tabular}
		\caption{Commands}
		\label{tab:commands}
	\end{table}
% section System Architecture (end)

\section{Methodology} % (fold)
\label{sec:Methodology}
	Our system enables feature set detection, minor feature restoration and
	quantitative analysis which can be decomposed into four stages. The
	first two stages detect feature sets: beads as features and then
	sprouts as features. In the third stage, the system attempts to restore
	a few of the sprouts by approximating connections between broken sprout
	segments. Finally, the system quantitatively analyzes the imaged assays
	through Sholl Analysis.

	\subsection{Bead Extraction} % (fold)
	\label{sub:Bead Extraction}
		Bead extraction is a two-step process. To reduce noise, the system
		first smooths the image using a Gaussian blur. Second, circles in
		the image are detected using the circular Hough Transform. The
		circles detected correspond to the beads in the assay.
		Subsequently, the origin and radius of the bead is obtained.
	% subsection Bead Extraction (end)

	\subsection{Sprout Extraction} % (fold)
	\label{sub:Sprout Extraction}
		Sprout extraction depends on bead extraction because the beads must be
		masked before sprout extraction occurs to separate beads from sprouts.
		We mask the beads given the geometry of the circles. Due to the
		disconnectivity of sprouts in the assay, we begin by obtaining all
		sprout segments and represent them as line segments. We distinguish the
		end points of each line segment for connectivity. The start point, $S$,
		is the end point on the line segment such that it is closer to the
		origin of the line segment's closest bead. The end point, $E$, is the
		end point on the line segment such that it is farther from the origin
		of the line segment's closest bead.

		To determine which line segments belong to the same sprout, we use
		euclidean distance between their start and end points enforced by
		the constraint that the end point must by closer to the origin of
		its closest's bead than the target start point. For example, two
		line segments are part of the same sprout if the distance between
		the start and end points are within a specified distance parameter,
		$d$.
	% subsection Sprout Extraction (end)

	\subsection{Sprout Restoration} % (fold)
	\label{sub:Sprout Restoration}
		The process of restoration is dependent on the successful
		extraction of HLSGs features from the imaged angiogenesis.
		Restoration is necessary due to the disjoint sprout segments that
		appear. Disjoint sprout segments can be caused by the an
		out-of-focus microscope. % subsection Sprout Restoration (end)

	\subsection{Sholl Analysis} % (fold)
	\label{sub:Sholl Analysis}
		Sholl Analysis is a quantitative method for quantitatively analyzing
		morphological characteristics of neurons. \cite{sholl53}
	% subsection Sholl Analysis (end)
% section Methodology (end)

\section{Results} % (fold)
\label{sec:Results}
	Display a comparison table with human counts.
% section Results (end)

\bibliography{prelim}
\bibliographystyle{plain}

\appendix
\section{Pseudo Code} % (fold)
\label{sec:Pseudo Code}
	This section outlines the pseudo-code for the Driver and HLSGExtractor
	operations. 

	\subsection{Driver} % (fold)
	\label{sub:Driver}
		The following pseudo-code outlines the Driver which reads input from the
		client, evaluates the input as a command, prints the output as a
		consequence of executing the command and then repeats this sequence of
		operations.
		\begin{algorithm}[ht!]
			\caption{Driver}
			\begin{algorithmic}
				\Procedure{Driver}{}
					\State running $\gets$ True
					\While{running}
						\State input $\gets$ read\_input()
						\State command $\gets$ parse(input)
						\State output $\gets$ HLSGExtractor.execute(command)
						\State print(output)
					\EndWhile
				\EndProcedure
			\end{algorithmic}
		\end{algorithm}

		Note that the driver executes while the \lstinline$running$ flag is
		true. Consequently, the REPL is responsible for setting this flag
		false.
	% subsection Driver (end)

	\subsubsection{Sprout Extractor} % (fold)
	\label{ssub:Sprout Extractor}
		Given an imaged assay and a set of bead features, the algorithm
		proceeds by masking the beads from the image. Next, a segmentation
		strategy is used to separate individual sprouts from the collection
		of globally detected sprouts. Finally, the segmentation strategy
		yields the detected feature set of sprouts.
		\begin{algorithm}[ht!] \caption{Sprout Extraction}
			\begin{algorithmic}
				\Procedure{ExtractSprouts}{img, beads}
					\State maskedImg $\gets$ maskBeads(img, beads)
					\State strategy $\gets$ SegmentStrategy(maskedImg,beads)
					\State sprouts $\gets$ strategy.segment()
					\State \Return sprouts
				\EndProcedure
			\end{algorithmic}
		\end{algorithm}
	% subsubsection Sprout Extractor (end)

	\subsubsection{HLSG Extractor} % (fold)
	\label{ssub:HLSG Extractor}
		\begin{algorithm}[ht!]
			\caption{HLSG Extraction}
			\begin{algorithmic}
				\Procedure{ExtractHLSGs}{img}
					\State beads $\gets$ ExtractBeads(img)
					\State sprouts $\gets$ ExtractSprouts(img, beads)
					\State hlsgs $\gets$ MapSproutsToBeads(sprouts, beads)
					\State \Return hlsgs
				\EndProcedure
			\end{algorithmic}
		\end{algorithm}
	% subsubsection HLSG Extractor (end)
% section Pseudo Code (end)


\end{document}
